\section{Mecânica do Jogo}


Nesta sessão descreveremos as regras de interação entre o jogador e o jogo.

\subsection {Mecânica Basica}

O jogador irá passar por mapas do jogo tendo a liberdade de andar em qualquer direção até que encontre um checkpoint que estará em uma região do mapa que não permite a volta do personagem a área anterior.

\subsubsection {Controles}

O jogador pode movimentar o personagem pelo cenário através das setas direcionais. Para as demais interações serão utilizadas as teclas ''A'', ''S'' e ''D''.
\begin{itemize}
\item ''D'' é o botão de ação, quando existem inimigos perto ele ataca, caso contrario ele realiza a interação com o cenário e outros personagens.
\item ''S'' é o botão de pulo, faz com que o personagem pule.
\item ''A'' é o botão de defesa, quando não for possível sair da frente de um ataque a tecla ''A'' pode ser usada para se defender.
\end{itemize}

\subsubsection {Deslocamento}

O personagem principal se desloca inicialmente correndo. Conforme este perde energia passa a correr mais devagar até que atinja um nível crítico
onde o personagem passa a andar apenas

\subsubsection {Inimigos e Desafios}

Nos mapas vão existir diversos inimigos que podem detectar o jogador se este chegar muito próximo 
e então dão inicio a uma perseguição, assim que o jogador se afasta a uma determinada distância esses voltam ao seu estado inicial. 

Estes inimigos quando próximos do jogador irão ataca-lo possibilitando que o jogador também faça o mesmo. 
Adicionalmente no final de cada mapa haverá um desafio extra como um inimigo mais forte ou uma batalha.  

\subsubsection {Interação com o Cenário}

Além dos inimigos o personagem principal poderá interagir com partes do cenário e outros personagens para completar os
desafios propostos.

\subsection {Combate}
\subsubsection{Inimigos}
Ao se aproximar dos inimigos, estes irão atacar o personagem principal. O personagem principal pode então pressionar a tecla ''D'' para atacar o inimigo mais 
próximo.

Os ataques dos inimigos serão periódicos podendo ser defendidos utilizando a tecla ''A'', que irá reduzir o dano causado, ou podem ser esquivados utilizando as 
setas direcionais para sair da região de efeito do ataque.

\subsubsection {Armas}
O personagem principal começa armado com um porrete de madeira. Durante a primeira fase este pode pegar uma lança que irá facilitar no combate com o tigre. 

Na segunda fase haverá um bastão em chamas que, entre outros usos, espanta os animais que estão próximos.

\subsection {Dificuldade}

O jogo tem apenas um modo de dificuldade. Os desafios propostos aumentam de dificuldade conforme a progressão dentro de cada fase.

De uma fase para outra são introduzidas novas mecânicas de jogo aumentando o grau de dificuldade e criando um incentivo para que o jogador 
não perca o interesse pelo jogo.

\subsection {Vide e Energia}

Existem duas barras principais durante o jogo, as barras de vida e energia. 

A barra de vida indica a saúde do personagem, esta barra diminui quando
o personagem sofre ataques dos inimigos e aumenta quando esse derrota os inimigos.

A barra de energia representa um desafio extra que depende da fase, esta barra diminui conforme o deslocamento do personagem e aumenta quando o 
mesmo cumpre alguns objetivos na fase.

\subsection {Salvar e Carregar o jogo}
?